\documentclass{article}


\usepackage{amsmath}
\usepackage{amsfonts}
\usepackage[margin = .5in]{geometry}
\usepackage{enumerate}

% Packages for Formatting Contents
\usepackage[utf8]{inputenc}


\title{Fixed Point Library Documentation}
\author{Michael J. Malek}
\date{August 28, 2017}

\begin{document}

\maketitle


\tableofcontents

\section{Introduction}

This library implements basic functions necessary for 32-bit fixed-point operations. 
This document explains the purposes  of all functions and their implementations. Supporting
files including ``parameters.h'' and ``fixedpoint.h'' are discussed as well. 

Model for the framework was drawn from the following source:

``code.google.com/archive/p/libfixmath/source/default/source''

OVERALL FEATURES:

\begin{enumerate}[i.]

\item 

\item Provides 

\end{enumerate}

\section{Parameters}




\section{Basic Arithmetic}

\subsection{fp32\_add}

Addition is the same as for regular integers.

\subsection{Subtraction: }

Subtraction is the 

\subsection{Multiplication}

There are three primary issues which can arise in our multiplication:

\begin{enumerate}[i.]

\item{Overflow}

\item{Underflow}

\item{Loss of Precision}

\end{enumerate}

The solution implemented deals with all three of these problems at once, 
and hinges upon the 

\textbf{Proposition.} Multiplication with double the precision guarantees
an exact answer

\textbf{Proof.} 


\subsection{Division}

\section{Basic Conversions}

\section{Saturating Arithmetic}

\section{Min/Max}

\section{Basic Exponential Functions}

\section{Basic Trigonometric Functions}





\end{document} 